% Created 2020-03-14 Sat 14:21
% Intended LaTeX compiler: pdflatex
\documentclass[11pt]{article}
\usepackage[utf8]{inputenc}
\usepackage[T1]{fontenc}
\usepackage{graphicx}
\usepackage{grffile}
\usepackage{longtable}
\usepackage{wrapfig}
\usepackage{rotating}
\usepackage[normalem]{ulem}
\usepackage{amsmath}
\usepackage{textcomp}
\usepackage{amssymb}
\usepackage{capt-of}
\usepackage{hyperref}
\author{David Pham}
\date{\today}
\title{Data analysis tools}
\hypersetup{
 pdfauthor={David Pham},
 pdftitle={Data analysis tools},
 pdfkeywords={},
 pdfsubject={},
 pdfcreator={Emacs 26.3 (Org mode 9.1.9)},
 pdflang={English}}
\begin{document}

\maketitle
\tableofcontents


\section{Introduction}
\label{sec:org439afed}

A friend ask me what I would recommend to learn if someone wanted to join the
field of quantitative finance, machine learning or statistics (the basic
skills are similar). He also asked to make some small remarks on why these
tools were relevant.

First of all, this is only a matter of personal taste and they are a lot of
reason why someone would learn and act differently, but my hope is that I can
provide some justification on why I invested time to learn these tools.

I personally use all the following tools on recurrent basis and I would
advise to really to learn the basics and use the programs that really help
you in your daily work. Obviously, this list does miss few
programs/languages, the reason is probably that I can not assert that I used
these other languages enough to have a decent opinion of them.

\subsection{General advice}
\label{sec:orgd6d2830}
If I may give only one advice, it would be the following:

\begin{quote}
DO NOT REINVENT THE WHEEL.
\end{quote}

Search hard for a solution of a problem, or try to shape a problem into a
solved problem. Even if you are the best scientist and coder, great
professionals almost always win time by only copy-pasting and understanding
solutions that already exists.  However, during your training, retry to
rewrite all the scripts that you learn: one learns about typos and subtle
mistakes by doing so.

\section{Quick overview}
\label{sec:org00bf32d}
This is a list of tools I genuinely recommend to learn and use in our daily
life when interacting with a computer.

A good point to remember about the suggestions is they originate from my
(little) experience and my concern about tractability of the analysis,
reproducible research, and productivity.

Let's get started!  Here is a short version:
\begin{itemize}
\item \textbf{Ubuntu}. Just do yourself a favor and work in an UNIX environment.
\href{https://ubuntugnome.org/}{Ubuntu Gnome homepage}
\item \textbf{Emacs} as your environment: write and code in any language without ever
without leaving your f/j keys in the same consistent environment.
\href{https://www.gnu.org/software/emacs/}{Emacs homepage} (or \href{http://vgoulet.act.ulaval.ca/en/emacs/}{here} for a more friendly-user version).
\item \textbf{Git} and \textbf{Github}: modern way for archiving, collaborating and working
with files.
\href{https://git-scm.com/}{Git homepage}
\item \textbf{Pandoc, markdown}: write content and not the formatting. Markdown and pandoc
automate most of the editing tasks.
\href{http://pandoc.org/}{Pandoc homepage}
\item \textbf{Python}: your first and most fun programming language. Learn to create
programs (not class) in a funny and interactive fashion.
\href{https://www.python.org/}{Python homepage}
\item \textbf{R}: your friend for visualizing and analyzing data.
\href{https://cran.r-project.org/}{R CRAN homepage} and \href{https://www.rstudio.com/}{RStudio IDE} (always handy when starting with R).
\end{itemize}

\section{Links}
\label{sec:org5437cbb}
\subsection{Emacs}
\label{sec:org0f2c551}
\begin{itemize}
\item Start with \texttt{alt-x help-with-tutorial}, to get the first introduction.
\item \href{http://ess.r-project.org/index.php?Section=download}{Emacs with R}
\item Some nice introduction from \href{https://www.youtube.com/channel/UCGM8KgUXqsS4d8-4rgWRWKg}{Udacity}.
\end{itemize}

\subsection{Data analysis}
\label{sec:org7308a96}
\begin{itemize}
\item \href{http://www.math.uwaterloo.ca/\~mhofert/contents/guidelines.pdf}{Guidelines for Statistical Projects: Coding and Typography}:  Great reading
and advice for writing \LaTeX{} documents and R Code.
\end{itemize}

\subsection{Git and Github}
\label{sec:orgafdf1f2}
\begin{itemize}
\item \href{https://www.udacity.com/course/how-to-use-git-and-github--ud775}{Gentle introduction} to git and github.
\item \href{https://udacity.github.io/git-styleguide/}{Udacity guideline} for editing git commits.
\item \href{https://git-scm.com/}{Git official website}
\end{itemize}

\subsection{Python}
\label{sec:org6f571e7}
\begin{itemize}
\item \href{https://www.udacity.com/course/intro-to-computer-science--cs101}{Best introduction ever} to python.
\item \href{http://www.scipy.org/}{Scipy}
\item \href{http://scikit-learn.org/}{Scikit}
\item \href{http://stanford.edu/\~mwaskom/software/seaborn}{Seaborn}
\end{itemize}

\subsection{R}
\label{sec:org290c779}
\begin{itemize}
\item \href{http://arxiv.org/pdf/1309.4402v1}{Simsalapar} documentation
\item \href{http://blog.rstudio.org/}{RStudio blog}: a good point for looking at best computer practices.
\end{itemize}

\subsection{Blog post}
\label{sec:org8605bae}
\begin{itemize}
\item \href{https://michaelochurch.wordpress.com/2012/07/27/six-languages-to-master/}{Language to master}, when beginning to learn software engineering.
\item \href{https://michaelochurch.wordpress.com/2012/08/15/what-is-spaghetti-code/}{Spaghetti code}, what to avoid in bigger projects.
\item \href{https://michaelochurch.wordpress.com/2013/01/09/ide-culture-vs-unix-philosophy/}{IDE vs Unix philosophy}, or why Emacs is awesome. Make small things, that do one thing well.
\item \href{http://www.joelonsoftware.com/articles/Unicode.html}{Unicode}: If you don't want to smash your head against the wall because of accent.
\end{itemize}

\section{Details}
\label{sec:org612103d}
The goal of this section is to provide a basic justification on why I am using
these programs or framework.

\subsection{\textbf{Ubuntu}}
\label{sec:org78b2e86}
Any Linux distribution would be fine.  However, I would recommend Ubuntu as
the first point of entry to the linux world. The advantage over Windows are
following.
\begin{itemize}
\item Open source software: Most advanced tools for data analysis rely on some
tools of the UNIX world and their installation is extremely simplified.
\item Ubiquitous terminal: interacting with several languages, tools is the
favored way of creating analysis (small modular pieces).  Using the shell
allow us to easily automate the boring tasks with shell scripts.
\item Remote computations: Most cloud computing services are based on a UNIX
distribution.
\end{itemize}

\subsection{Text editor/IDE: \textbf{Emacs}}
\label{sec:orgb956bed}

This topic is probably one of the most controversial.  Most of my work goes
around reading and writing. I am unfortunately paid to write and authors
things: code for analysis, automating tasks, formatting or cleaning data,
and last but not the least, writing documentations and reports.  One of the
source of inefficiency when writing documents is the time we move our hands
from the keyboard to the mouse in order to move the cursor around a document
or characters.  Emacs provide a rich amount of shortcuts to help the user to
write everything without leaving the keyboard and it also allows to
completely modify them as well.
\begin{itemize}
\item Truly personal editor: One of the most interesting feature is also
that one can really suits Emacs to its need. For example, I have a Swiss
keyboard, and deleting backwards is a task I often have to do. However,
with my keyboard layout, I have to move little finger to backspace, which
disturb my right hand position. Hence, to solve this problem, I simply
bound \emph{CTRL-é} and \emph{ALT-é} to \texttt{delete-char-backwards} and
\texttt{delete-word-backward}.
\item One editor for all languages: With Emacs, you will develop code (python,
R, C++ and clojure are supported), edit text (\LaTeX{} and Markdown) and use
the shell, all with a consistent set of shortcuts that you set.
\end{itemize}

Note that I do not claim, it will be easy at the beginning. However, after a
few days, Emacs will save you thirty minutes to a hour each day!

\subsection{Document generation: \textbf{Pandoc} (markdown, \LaTeX{})}
\label{sec:org45fb12c}
Stop using Word or Libre Office Writer. Just stop it right now!  The flaw
with these document editors is the user has only one interface with the
program for two distinct tasks: content creation and formatting.

Users often want standard format and focus on the content. \LaTeX{} and
markdown are markup language that abstract the formatting process from the
user. Then, the computer takes care of the references, the layout of
listings, the exact amount of spaces around words.

\textbf{Markdown} is simplified version of \LaTeX{} (but still really capable) and
\textbf{pandoc} is a program that can freely convert your markdown or \LaTeX{} file
into pdf, html or even docx.

A good argument in favor of these pure text file is that version control
come almost for free and UNIX diff functions (to display difference between
documents) works seamlessly.


\subsection{Programming languages for data analysis:  \textbf{python} and \textbf{R}}
\label{sec:orge4e3823}

There is no clear winner between the two languages when analyzing data and
each of them has its own advantages. Both provide great advantages:

\begin{itemize}
\item REPL (interactive console for coding);
\item Open source;
\item A wonderful open source community;
\item An uncountable set of tutorials for beginning;
\item Object-oriented and functional programming paradigm;
\item Easy interaction with most used compiled languages (C, C++, Java).
\item Exceptional documentation for the base package.
\end{itemize}

Advice: if you have little experience in coding, I totally support learning
\textbf{python} first and then quickly pick up with \textbf{R}, because \textbf{python} also
emphasis data structures (aka speed of your code) and software engineering
(maintainability and reliability).

Main differences are emphasized in the next subsections. Note, should you
choose to use \textbf{python}, please try to learn \textbf{python 3} instead of 2.7 as it
will be depreciated in 2020, and no 2.8 is planed.

\subsubsection{Python}
\label{sec:orgb666a9b}
Python is often favored by the machine learning community.
\begin{itemize}
\item All-battery included programming languages: there is a module for almost
everything in the base installation (csv, itertools, regular expression).
\item Explicit data structure (list, tuple, set, frozenset, hash-map/dictionary).
\item There is only ONE true way of writing code (so most people tend to have
the same style).
\item It has one of the most beautiful syntax among programming languages.
\item There is a default support for modules (i.e. writing independent code in
other files and reusing them) and it is trivial to use (\texttt{import
      my\_other\_file}, wher \texttt{my\_other\_file.py} contains what you want to reuse).
\end{itemize}

\subsubsection{R}
\label{sec:org9ae7172}
R is the standard programming language for most data analysts.
\begin{itemize}
\item Almost any statistical learning algorithm has already been implemented
and is available on CRAN.
\item Installation of new packages is a lot easier, when the user has no \textbf{admin
rights}. This is really important as most companies do not provide them
to their employees.
\item Programming language created by statisticians for statisticians.
\item Syntax is really targeted for performing data analysis.
\item Fairly high level language. Packages should take care of most system error
(segmentation fault or out of bound memory).
\item Amazing data visualization capabilities.
\item Interactive data visualization with javascript is definitively more
advanced than with python.
\item \emph{Vectorized} operations are favored to \texttt{for} loop.
\end{itemize}

\subsection{Version control: \textbf{git} and \textbf{github}}
\label{sec:org8ce5d97}
I have honestly used them since only recently. Hence my opinion and my
experience might not be really significant. But version control definitively
helps. It avoids having archive folders everywhere, and it tracts the
development of our work.
\subsection{Finally, if you want to get a corporate job}
\label{sec:org656e782}
\subsubsection{SQL for databases}
\label{sec:org3c98f41}
I am honestly not a specialist of databases, especially as I prefer simple
format such as csv. But as data grow larger, it is clear that database have
advantage. I suppose we will only remain user, so it does help to know the
basics of SQL. From there, a lot of language are similar but if one has to
write queries often, some packages from python or R will help to not
become crazy.
\subsubsection{Excel, VBA}
\label{sec:orgdbf667e}
Excel and VBA are the most dangerous tools in the corporate world. Anyone
can make extremely advanced data analysis and plots. However, it is almost
impossible to reproduce the research and to avoid mistakes. Hence my advice
is that, if possible, try everything to avoid using with Excel files:
\begin{itemize}
\item Use some python module to interact with Excel.
\item Create vba macros to export/import data in \texttt{.csv} and call a python
script with the \texttt{Shell} vba function to work with the data.
\end{itemize}

If anyhow, you still MUST use Excel, here are some tricks:
\begin{itemize}
\item Excel
\begin{itemize}
\item Pivot table allows to create simple "*-by" summary of your data
(e.g. "sum-by region", "count-by country").
\item Filter function and deleting duplicate elements are great built-in
tools.
\item \texttt{vlookup}: the best way to reproduce the behavior of hash-map in Excel.
\end{itemize}
\item VBA
\begin{itemize}
\item Record macro is your best friend.
\item Turn off display updating when running macros.
\item If possible, work with arrays in memory and not cells on the
spreadsheet.
\item Use Excel functions whenever possible.
\item Include Microsoft.Scripting module to get hash-maps (associative
arrays).
\item When filtering data, use the \texttt{filter} function from Excel with the
correct parameters.
\end{itemize}
\end{itemize}

\section{Intermediate topics}
\label{sec:orgd859b7e}
I consider here some intermediate topics

\subsection{Ubuntu}
\label{sec:org58ce54b}
I don't have much to add about which OS you should use. You can argue that
most paid job require some knowledge of C\# and the .Net platform.
Nonetheless, most open source tools have been thoroughly tested on Linux
machines, hence you will avoid some subtle bugs by adopting Ubuntu.

\subsection{Emacs}
\label{sec:orgb8de454}

I would like use the same tool at work and at home, as I want to be able to
improve my skills continuously, especially when I am working (and we do use
Emacs at my working place now).  The alternative solution is to use the
IDEs.  Nevertheless, most of them are quite complicated (which is also true
for Emacs) and the possibilities of customizing them are quite limited.

Moreover, using a bare text editor force professionals to understand the
engineering process behind their products when compiling.

One of the best argument, I have heard about using Emacs are the following:
\begin{itemize}
\item Emacs has been here for the past 30 years, and will probably still be
there in the next 30 years. One of the reason why it could disappear is
the non-integration of webkit (but we will figure out how to make it).
\item True integration with UNIX tools: You will find yourself using grep,
find, top without knowing.
\item A wonderful way to learn to code as well.
\item Most importantly: the feeling of mastery. Every day, I learn something
new about Emacs and I am sure I can use it in my workflow. Otherwise I
learn something about elisp (which is never bad). Overall, it made me a
better programmer as well as it gave me the feeling about being able to
tackle all challenge with a common set of tools.
\end{itemize}

\subsection{Document generation}
\label{sec:org96aa316}
A good option in combination with Emacs is to use \textbf{Org-mode} for \textbf{literate
programming}. Org-mode is an equivalent of markdown (and is also supported
by pandoc). Nevertheless, it provides a better integration with Emacs and
its default HTML and \LaTeX{} output hare nicer. Bonus: it is a really good
TODO list remainder.

Literate programming is also an interesting topic: usually, programmers
write code, filled with comments, but this could or should be
reverse. Authors should write text with code blocks describing their
views. Org-mode totally support this type of document. For a good
alternative, search for \texttt{rmarkdown}.

\subsection{Python}
\label{sec:orgad0ab13}
The \textbf{Scipy} stack includes most of what you will ever need to replace
matlab: numpy (approximately matrix calculation), scipy (optimization and
numerical mathematics), ipython (a better repl/interactive console), pandas
(data.frame like data structure), simpy (symbolic mathematics) and nose (for
testing). \textbf{pandas} is the most important package to learn for data analysis.

\textbf{Scikit-learn} offers most machine learning algorithm for statistical
prediction. \textbf{Seaborn} is a great abstraction over matplotlib for plotting
*data.

\subsection{R}
\label{sec:orga4ad457}
There are several types of R packages: packages providing statistical
models, others providing infrastructures for performing analysis. I will
give you some advice for the second type of packages.

\textbf{simsalapar} abstract and provide a great framework for computing simulation
over grids of parameters.

\textbf{data.table} is probably the most interesting packages. It offers a C++
implementation of the data.frame data structure and provides incredible
speed up. The \texttt{fread} function also accelerate incredibly the loading of
\emph{csv} data. One of the drawback is the way it uses expression and is maybe
not the most natural way for R programmers to use it.

\textbf{ggplot2} and \textbf{lattice} make the plotting experience of R even
sweeter. lattice gives quick visualization for faceting (e.g. xy-plot for
\emph{x} against \emph{y} grouped by \emph{z}), whereas ggplot2 wraps the \textbf{grid} package to
make highly customized data visualization. lattice is often faster than
ggplot2, but ggplot2 is maintained and its output can customized.

\textbf{caret} abstracts the best practices of data analysis to allow more
flexibility. It offers a framework to compare and average models and perform
the split of data (test-training-evaluation set) automatically.

\textbf{parallel} come by default in R. I suggest to learn how to write parallel
code with clusters and not with multiple cores. The reason is that forking
copies completely your master thread which could cause some problem if you
have a finite amount of RAM.

\textbf{timeDate} from the RiskMetrics team is a really appreciable abstraction of
calendar issue in the financial world.

\textbf{regular expression} (or \textbf{regex}) are supported by default in R, learn about
them, they are life savers!

\textbf{magrittr} provides the pipe operator \texttt{\%>\%}. More or less, it allows to read
expression from left to right, compared to inside to outside:
\begin{verbatim}
x %>% f %>% g(y) == g(f(x), y) # TRUE
\end{verbatim}

\subsection{Statistics: Repeat the basics}
\label{sec:org1c0313d}
I love shiny statistical learning methods, I mean, I do love them (neural
networks, lasso/ridge regression, random forest, copula modeling). However, I
am convinced that for mastery, one do have to repeat the basics often and to
have solid understanding of what \emph{randomness} means in statistics.

\section{Things that exists which I never used professionally}
\label{sec:org385b151}
As any curious person, I try to surf on the net and see what could be
interesting to add to my set of skills. Unfortunately, I could never
professionally use the tools which I am describing below, but I wanted to
point at them for the sake of completion.

\subsection{Linear Algebra}
\label{sec:orgd2850f4}
Matrix notation is certainly helpful. Nevertheless, for advanced numerical
computations and optimization, \emph{vectorization} (using values locations in
memory to perform calculation) is primordial. I unfortunately never had to
deal with such constraints, but it would not hurt to have an overview about
LAPACK, BLAS and ATLAS.

\subsection{C, C++ and Cuda}
\label{sec:org3c7abdc}
\textbf{C} and \textbf{C++} are almost the \emph{de facto} way to improve R or python code and
\textbf{Cuda} gives access to the power of your graphic card for computing.
I do not have a strong opinion of them, I may be too inexperienced.

What I do know is that for short function, these language are really
manageable and quit elegant as well.

\subsection{Java (JVM), Clojure, Scala and Jython}
\label{sec:orgf4c7e19}
No one would argue that Java is used everywhere. However, the language has
its constraints on its syntax that makes it quite not enjoyable.

I really appreciate \textbf{Clojure} for its simplicity and for its tradition for
Lisp, however, one will more easily find a job with \textbf{Scala}.

As last resort, \textbf{Jython}, the JVM implementation of python, can also help.

\subsection{Javascript and Clojurescript}
\label{sec:org4996a83}
Interactive data visualization is the future and your browser is probably
one of the best tool to display data. \textbf{D3.js} is one of the best framework
and \textbf{metricsgraphics.js} wraps around it in order to abstract some of the
complexities.

Clojure is a wonderful lisp dialect, and Clojurescript uses the same
semantics, but targets the Javascript instead of the JVM. Even though I am
still in the process of experimenting Clojurescript, I am certain that it
will enhance the maintainability and performance of my code.

\subsection{Python}
\label{sec:org25a75c5}
\textbf{Cython} seems to be a nice start for optimizing your python code. If you
really want to go into new fields, \textbf{Tensorflow} from Google is a good start.

\subsection{R}
\label{sec:orge4749c0}
\textbf{Rcpp} is a wonderful way to interact R with C++.
\end{document}
